% Criba de Eratóstenes
% Ejemplo de como usar código en lua para generar un documento tex
% Procesar con lualatex

\documentclass[a4paper]{article}

\usepackage[spanish]{babel}
\usepackage{luacode}
\usepackage{xcolor}
\usepackage{amsmath}

\newcommand{\hlightP}[1]{%
  \ooalign{\hss\makebox[0pt]{\fcolorbox{red!30}{green!40}{$#1$}}\hss\cr\phantom{$#1$}}%
}
\newcommand{\hlightC}[1]{%
  \ooalign{\hss\makebox[0pt]{\fcolorbox{green!30}{red!40}{$#1$}}\hss\cr\phantom{$#1$}}%
}


\begin{document}

% http://wiki.luatex.org/index.php/Writing_Lua_in_TeX

%https://wiki.contextgarden.net/Programming_in_LuaTeX

\title{Criba de Eratóstenes}
\author{Pablo De Nápoli}
\maketitle

\begin{luacode}
function mostrar_criba(n,criba)
    columnas=10
    tex.print("$$")
    tex.print("\\begin{pmatrix}")  
    for i = 1, n do
  
    if criba[i] then
      tex.sprint("\\hlightP")   
    else
      tex.sprint("\\hlightC")      
    end
    tex.sprint("{")
    tex.sprint(i)
    tex.sprint("}")
    if i % columnas== 0 then 
      tex.sprint(" \\\\")
    else 
      tex.sprint(" & ")
    end
  end
tex.print("\\end{pmatrix}")  
tex.print("$$")
end 

--- n representa hasta donde queremos calcular la tabla de primos

local n = 100
local criba = {}
local primos = {}

--- incialemnte ningún número fue tachado en la criba
criba[1]=false
for i = 2, n do criba[i] = true end

tex.print("Criba inicial")
mostrar_criba(n,criba)
for i = 2, n do
 if criba[i] then
   -- encontramos un primo, lo agregamos a la lista
   primos[#primos+1] = i
   -- tachamos sus múltiplos, son compuestos
   tex.print("tachamos los múltiplos de ", i,"\\\\")
   tachamos_alguno=false
   for j = i*2, n, i do 
      criba[j] = false
      tachamos_alguno=true 
    end
    if tachamos_alguno then
       mostrar_criba(n,criba)
    end
 end
end

tex.print("\\\\ Primos encontrados:")
-- Muestra la lista de primos en el documento latex
tex.sprint(table.concat(primos, " "))

\end{luacode}


\end{document}


